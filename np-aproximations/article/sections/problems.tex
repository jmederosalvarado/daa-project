\documentclass[../np-approximations.tex]{subfiles}
 
\begin{document}

\section{Problems}

\subsection{Approximating Degree-MST}

\begin{statement}
	Dado un grafo $G=(V,E)$ el problema consiste en encontrar un árbol abarcador de costo mínimo, tal que el degree de su vértice con mayor degree sea mínimo.
\end{statement}

\begin{definition}
	Dado un árbol de cubrimiento $T$ en un grafo $G$, considere una 
	arista $(u,v)$ de $G$ que no está en $T$, y sea $C$ el único 
	ciclo que generaría dicha arista de ser añadida a $T$. Suponga 
	que existe un vértice $w$ en $C$ con la propiedad de que
	$\rho(w) \ge \max(\rho(u),\rho(v))+2$, donde $\rho(x)$ es el 
	degree de un vértice $x$ en $T$. Denótese como mejora local al 
	proceso de añadir a $T$ la arista $(u,v)$ y eliminar cualquiera 
	de las otras aristas de $T$ incidentes en $w$, dado que
	$\max(\rho(u),\rho(v),\rho(w))$ ha sido decrementado.
\end{definition}

\begin{definition}[LOT]
	Un árbol localmente óptimo (\emph{LOT}), es un árbol en el cual 
	ninguna de las aristas que no pertenecen a él puede producir 
	\emph{mejoras locales}.
\end{definition}

Se denotan $T^*$, $\Delta^*$ como el \emph{MST} óptimo, y su degree 
maximal.Sea $k$ el degree maximal de un \emph{LOT} $T$, se define 
$S_i$ como el conjunto de vértices de degree al menos $i$ en $T$.

\begin{theorem}
	Para cualquier $b>1$, el degree maximal $k$ de un árbol 
	localmente óptimo $T$ es menor que
	$(b*\Delta^* + \ceil{\log_b n} )$. Por tanto
	$k \in O(\Delta^* + \ceil{\log n} )$
\end{theorem}

\begin{proof}
	Note que para $i=k,k-1,\dots$, el factor de expansión de los 
	conjuntos $S_i$ ($\abs{S_{i-1}}/\abs{S_i}$) puede ser más 
	grande que $b$, a lo sumo $\log_b n$ veces seguidas, es decir, 
	existe $i$, tal que $k-\ceil{\log_b n} + 1 \le i \le k$ y
	$\abs{S_{i-1}} \le b \abs{S_i}$. De lo contrario,
	\begin{equation}
		\label{eq:ratio}
		\abs{S_{k-\ceil{\log_b n}}} \ge
		b^{\ceil{\log_b n}} \abs{S_k}
	\end{equation}
	lo cual es una contradicción, dado que $\abs{S_i} \le n$, 
	mientras que $b^{\log_b n} \ge n$ y $\abs{S_k} \ge 1$.
																                                    
	Suponga que se eliminan los vértices de $S_i$ de $T$. Esto 
	dividiría $T$ en un bosque $F$ con $t$ árboles. Como cada 
	vértice en $S_i$ tiene degree al menos $i$, hay al menos
	$i\abs{S_i}$ aristas incidentes a estos vértices. También como 
	$T$ es un árbol, a lo sumo $\abs{S_i} - 1$ estan en $S_i$, y 
	por tanto son contadas dos veces, por tanto,
	\begin{equation}
		\label{eq:t.lb}
		t \ge i\abs{S_i} - 2(\abs{S_i} - 1)
	\end{equation}
																                                    
	Observe que toda arista entre árboles distintos de $F$ es 
	incidente en al menos un vértice de degree $i-1$. Sea $e=(u,v)$ 
	una arista que cruza entre dos árboles de $F$. Entre $u$ y $v$ 
	existía un camino en $T$, dado que $u$ y $v$ estan en árboles 
	distintos de $F$, existe al menos un vértice $w_0$ en el camino 
	de $u$ a $v$ tal que $w_0 \in S_i$. Luego por la condición de 
	optimalidad de los \emph{LOT},
	\begin{equation}
		\begin{split}
			\rho(w_0) & < \max(\rho(v),\rho(u)) + 2 \\
			\rho(w_0) & < \max(\rho(v),\rho(u)) + 2 \\
			i - 2     & < \max(\rho(v),\rho(u))     
			\quad \text{dado que $\rho(w_0) \ge i$} \\
			i-1       & \le \max(\rho(v),\rho(u))   
		\end{split}
	\end{equation}
	Han sido encontrados entonces $t+\abs{S_i}$ componentes, tal 
	que cada arista entre componentes es incidente en al menos un 
	vértice de $S_{i-1}$. En cualquier árbol abarcador de costo 
	mínimo existen al menos $t+\abs{S_i}-1$ aristas conectándo 
	estas componentes y cada una de estas aristas es incidente en 
	al menos un vértice en $S_{i-1}$. Luego en cualquier árbol 
	abarcador de $G$, el degree promedio de los vértices de
	$S_{i-1}$ es al menos,
	\begin{equation}
		\frac{t+\abs{S_i}-1}{\abs{S_{i-1}}}
	\end{equation}
	El degree maximal $\Delta^*$ es al menos el degree promedio de 
	estos vértices. Luego la ecuación  implica,
	\begin{equation}
		\label{eq:opt.lb}
		\begin{split}
			\Delta^* & \ge \frac{t+\abs{S_i}-1}{\abs{S_{i-1}}}   \\
			& \ge \frac{(i-1)\abs{S_i} + 1}{b\abs{S_i}} 
			\quad \text{por \eqref{eq:t.lb} y \eqref{eq:ratio}}\\
			& > \frac{i-1}{b}                         
		\end{split}
	\end{equation}
																                                    
	Combinando \eqref{eq:opt.lb} con la condición en el rango de 
	$i$, se obtiene,
	$$k < b\Delta^*+\ceil{\log_b n}$$
\end{proof}

Note que solo fue utilizada la condición de optimalidad para los 
vértices con \emph{alto} degree, luego, este resultado se sostiene 
para cualquier árbol que satisfaga la condicion de optimalidad 
local para los vértices en $S_i$ con $i = k - \ceil{\log_b n} + 1$.
Pudiera llamarse a estos árboles, árboles pseudo optimales
\emph{POT}.

El siguiente algoritmo resuelve el problema en tiempo polinomial.
Se comienza con un árbol abarcador arbitrario $T$ de $G$. Sea k el 
degree máximo de un vértice de $T$. En cada fase el algoritmo trata 
de reducir el degree de algún vértice de $T$ cuyo degree está entre 
$k$ y $k-\ceil{log n}$, utilizando la técnica de mejoras locales. 
El algoritmo se detiene cuando ningún vértice en $S_i$ puede ser 
mejorado de la manera antes descrita.

\subsection{Edge-Coloring}

\begin{statement}
	Dado un grafo $G=(V,E)$. Se desea determinar la menor cantidad de
	colores, con los que se pueden colorear las aristas del mismo, de 
	manera tal que en un vértice no coincidan nunca dos aristas del
	mismo color.
\end{statement}

Sea $d$ el degree del vértice de mayor degree en el grafo $G=(V,E)$.
Sea $C$ la menor cantidad de colores que se necesitan para colorear 
las aristas de manera válida.

Es fácil notar que, $C \ge d$, dado que si existe un vértice en el 
que inciden $d$ aristas y se utilizan menos de $d$ colores, dos de 
estas aristas tendrían el mismo color, invalidando así esta 
coloración.

Es posible demostrar también que $C \le d+1$, teorema de Vizing. 
Note que este resultado permite obtener una aproximación $C$, que 
no diste en más de una unidad.

La demostración del teorema anterior se realizará obteniendo un algoritmo
que dado un grafo con una coloración aceptable para sus aristas y una única
arista sin colorear, devuelve una coloración válida de dicho grafo con la
arista ya coloreada. Note que luego de construido este algoritmo, utilizar
el mismo para obtener una coloración aceptable de las aristas con a lo sumo
$d+1$ colores.

\bigskip
\begin{tcolorbox}
	\textbf{Algoritmo (Color One Edge)}
	\begin{enumerate}
		\item Sea $xy$ la única arista no colorada.
		\item Mientras el \emph{c(y)-camino} termine en $y$:
		      \begin{enumerate}
		      	\item Determine $z$ tal que la arista $xz$ tiene color $c(y)$.
		      	\item Asigne a $xy$ color $c(y)$ y descoloree la arista $xz$.
		      	\item $y \leftarrow z$.
		      \end{enumerate}
		\item Intercambie los colores en el \emph{c(y)-camino}.
		\item Asigne a la arista $xy$ color $c(y)$.
	\end{enumerate}
\end{tcolorbox}
\bigskip

La siguiente notación es utilizada de aquí en adelante:
\begin{itemize}
	\item $acc(v)$, predicado que se hace verdadero si la 
	      coloración de $v$ es aceptable, es decir, no existen dos 
	      aristas del mismo color incidentes en $v$.
	\item $c(v)$ es un color asignado a $v$
	\item $f(v)$, predicado que expresa que no existe ninguna
	      arista de color $c(v)$ incidente en $v$.
\end{itemize}

Se introduce también el concepto de \emph{camino alternante} que 
sera necesario para la demostración. Este concepto se refiere a un 
camino maximal que alterna entre, a lo sumo, dos colores dados. Es 
posible obsevar lo siguiente acerca de dichos caminos,
\begin{enumerate}
	\item Un camino alternante es un ciclo o un camino 
	      simple.
	\item Un par de colores y un vértice determinan un único 
	      camino alternante (se permite que ambos colores sean iguales, 
	      en cuyo caso la longitud del camino es a lo sumo 1).
	\item Intercambiar los colores de las en un camino 
	      alternante mantiene la coloración aceptable, dado que los 
	      camino alternantes son maximales, lo cual permite intercambiar 
	      los colores sin afectar el conjunto de colores en los vértices 
	      extremos.
\end{enumerate}

Se hará referencia a $(\forall_{v \in V} acc(V))$ como \emph{(0)}, 
y a \emph{(xy es la única arista no coloreada)} como \emph{(1)}.

Por \emph{(2)} se referirá a la siguiente proposición. âra cualquier color $p$
tal que el \emph{p-camino} es no vacío, sea $xs$ la primera arista de color $p$
en el camino, y $l$ el último vértice de este camino, entonces,
$c(l)=p \implies f(s)$.


El algoritmo anterior se encarga de, dado un grafo donde todas las 
aristas están coloreadas de manera aceptable y una única arista sin 
colorear, devolver una coloración de dicho grafo donde todas las 
aristas estan coloreadas de manera aceptable.

El grafo que recibe el algoritmo cumple $(0)$, $(1)$ y
$(\forall_{v \in V} f(v))$. Dado que $xy$ es la única arista no 
coloreada y el grafo no tiene lazos, $x \neq y$. Como
$(\forall_{v \in V} f(v))$, entonces se cumple $f(x)$, $f(y)$ y
$(2)$, Luego se puede afirmar,
\begin{equation*}
	(0) \wedge (1) \wedge (2) \wedge
	f(x) \wedge f(y) \wedge x \neq y
\end{equation*}

Si se entra al ciclo, es decir el \emph{c(y)-camino} termina en $y$,
como $x \neq y$ es posible afirmar que el \emph{c(y)-camino} 
comienza con una arista de color $c(y)$. De lo anterior se tiene 
que existe un $z$ único tal que $xz$ tiene color $c(y)$, en este 
caso es la primera arista del \emph{c(y)-camino}, dado que no 
existen lazos, $x \neq z$. $f(z)$ se deriva de $(2)$ con $p=c(y)$,
$s=z$, $l=y$. Se cumple también $y \neq z$, de lo contrario la 
arista $xy$ tendría color $c(y)$, lo cual es imposible.

Luego de darle a $xy$ color $c(y)$ y a $xz$ quitarle el color.
Se desea obtener,
\begin{multline*}
	(xz \text{ es la única arista no coloreada}) \wedge
	(xy \text{ tiene color } c(y)) \wedge \\
	(0) \wedge (2) \wedge f(x) \wedge \neg f(y) \wedge f(z) \\
\end{multline*}
Los primeros dos términos de la expresión anterior son verdaderos 
debido a que precisamente la intrucción que se ejecuta se encarga 
de ello. Para los siguientes tres términos obsérvese que solo es 
necesario analizar $x$, $y$, y $z$ dado que son los únicos 
afectados por cambios de colores. Los colores incidentes en $x$ se 
mantienen los mismos, luego $acc(x)$ y $f(x)$. Los colores 
incidentes en $z$ disminuyen, luego tambien $acc(z)$ y $f(z)$. A 
los colores incidentes en $y$ se agrega $c(y)$, como anteriormente
$f(y)$ entonces se mantiene $acc(y)$ y se obtiene $\neg f(y)$. Para 
probar que $(2)$ no cambia su veracidad consideremos dos casos.

Si $p \neq c(y)$, obsérvese que la instrucción de colorear y remover
color, mantiene el camino idéntico; en particular, si el camino es 
no vacío, sus vértices $s$ y $l$ se mantienen los mismos. La no 
variación de $l$ asegura la no variación de $c(l) \neq p$. Observe 
que $f(z)$ se mantiene, dado que como el único vértice cuyo valor 
de $f$ varía es $y$, y como $xs$ tiene color $p$, $xy$ no tiene 
color, y no existen aristas múltiples, $x \neq y$.

Si $p = c(y)$ obsérvese que el camino $x,z,\dots,y$ es reemplazado 
por $x,y,\dots,z$, luego tendría que demostrarse $(2)$ para
$p=c(y)$, $s=y$, $l=z$, es decir $f(y) \vee c(y) \neq c(z)$, de lo 
cual ya se tenía anteriormente $c(y) \neq c(z)$ y dado que $c$ no 
se modifica, esto se mantiene.

Al llegar a la ejecución de la primera instrucción después del ciclo
es posible garantizar que se cumple $(0)$, $(1)$, $f(x)$, $f(y)$ y que el
\emph{c(y)-camino} no termina en $y$. Al intercambiar los colores del
\emph{c(y)-camino}, se matiene $(0)$ por la \emph{propiedad 3} de los caminos 
alternantes. Dado que el camino no termina en $y$, entonces luego de intercambiar 
los colores del mismo, no se afecta $f(y)$, luego se se cumple $f(y)$. Si antes de 
la ejecución de esta instrucción, en $x$ no incicdía $c(y)$ entonces esto se 
mantiene, dado que el \emph{c(y)-camino} tendría longitud 0, no afectándose así la 
coloración de las aristas; de lo contrario, si $c(y)$ incidía en $x$ entonces el 
intercambio de colores cambia el color a esta arista por $c(x)$, por tanto ya $c(y)$ 
no incidiría en $x$. Esta instrucción no afecta la coloración de la arista $xy$. Una 
vez en este punto es posible asignarle color $c(y)$ a la arista $xy$, dado que como 
se observó anteriormente, se cumple $f(y)$ y $c(y)$ no incide en $x$, se mantiene
$(0)$. Luego, como era $xy$ la única arista sin colorear, al terminar la ejecución 
del algoritmo se puede garantizar que todas las aristas fueron coloreadas de manera 
aceptable.

\begin{theorem}
	El algoritmo anterior tiene costo $O(n)$.
\end{theorem}

\begin{proof}
	Obsérvese que el costo del algoritmo anterior esta dado por el costo del ciclo.
	En cada iteración de este, la cantidad de valores verdaderos que tiene el predicado $f$ disminuye en $1$. Luego de $n-1$ iteraciones, solo un vértice
	evaluaría de verdadero en el predicado. Dado que se cumple en todo momento durante el ciclo que $f(x)$ es verdadero y se cumple \emph{(2)}, entonces en la n-ésima iteración, $\neg f(s)$ para $s$ primer vértice despues de $x$ en el \emph{c(y)-camino}, aplicando \emph{(2)} $c(l) \neq c(y) \implies l \neq y$. No cumpliendose así la condición de repetición del ciclo.
\end{proof}

\subsection{Steiner Tree}

\begin{statement}
	Dado un grafo no dirigido $G=(V,E,w)$ donde $w$ es una función
	asigna un costo a cada arista, y un conjunto $S\subset V$. Se desea encontrar el árbol de menor costo que abarque todos los vértices de $S$.
\end{statement}

Dado un grafo no dirigido $G=(V,E,w)$ y un conjunto $S\subset V$. Considérese el grafo $G_1=(V_1,E_1,w_1)$ construído a partir de $G$ y $S$, de tal manera que $V_1=S$
y, para cada $\{v_i,v_j\}\in E_1$, $w(\{v_i,v_j\})$ es igual al costo del camino menos costoso desde $v_i$ a $v_j$ en $G$.

\bigskip
\begin{tcolorbox}
	\textbf{Algoritmo (Steiner Tree)}
	\begin{enumerate}
		\item Construya el grafo $G_1$ como se define anteriormente, a partir de $G$ y $S$.
		\item Encuentre un árbol abarcador de costo mínimo, $T_1$, de $G_1$.
		\item Construya el subgrafo, $G_s$ de $G$ reemplazando cada arista en $T_1$ por su correspondiente camino mas corto en $G$.
		\item Encuentre un árbol abarcador de costo mínimo, $T_s$, de $G_s$.
		\item Construya el Steiner Tree $T$ a partir de $T_s$, a partir de eliminar aristas en $T_s$ tal que toda hoja sea un vértice de $S$.
	\end{enumerate}
\end{tcolorbox}
\bigskip

Sea el grafo no dirigido $G=(V,E,w)$, y el conjunto de puntos 
$S\subset V(G)$ la entrada del algoritmo anterior. Sea $W$ la el 
costo del \emph{Steiner Tree}, $T$ producido por el algoritmo 
anterior. Sea $W_{min}$, el costo total del \emph{Steiner Tree} 
óptimo.

\begin{lemma}
	Sea $T$ un árbol con $m \ge 1$ aristas. Entonces, existe un 
	ciclo en $T$, $u_0, u_1, \dots, u_{2m}$ donde cada $u_i$,
	$1 \le i \le 2m$, es un vértice en $T$, tal que,
	\renewcommand{\theenumi}{\roman{enumi}}
	\begin{enumerate}
		\item Toda arista de $T$ aparece exactamente dos veces en 
		      el ciclo.
		\item Toda hoja de $T$ aparece exactamente una vez en la 
		      secuencia y si $u_i$, $u_j$ son dos hojas en el ciclo,
		      con niguna otra hoja entre ellas, entonces,
		      $u_i, u_{i+1}, \dots, u_j$ es un camino simple.
	\end{enumerate}
\end{lemma}

\begin{proof}
	Se probará el lema anterior mediante inducción en $m$. Para 
	$m=1$, sea $\{u_1, u_2\}$ la arista de $T$. Considere el ciclo
	$u_1,u_2,u_1$, este satisface (i) y (ii) en el lema. Asuma que 
	para $m=k$ el lema se cumple. Entonces para $m=k+1$, sea $v_p$ 
	una hoja de $T$ y $\{v_p,v_q\}$ una arista incidente en $v_p$, 
	considere el árbol $T'$ formado al eliminar la arista
	$\{v_p,v_q\}$ y el vértice $v_p$ de $T$. $T'$ es un árbol de 
	$k$ aristas, dado que $v_p$ era una hoja. Luego, por hipótesis 
	de inducción existe el ciclo $u'_1,u'_2,\dots,u'_{2k}$ en $T'$ 
	que satisface (i) y (ii). Es fácil ver que al reemplazar la 
	primera ocurrencia de $v_q$ por $v_q,v_p,v_q$, se cumple el 
	lema para $T$.
\end{proof}

\begin{theorem}
	Sea el grafo no dirigido $G=(V,E,w)$, y el conjunto de puntos 
	$S\subset V(G)$ la entrada del algoritmo anterior. Sea $W$ el 
	costo del \emph{Steiner Tree} $T$, producido por el algoritmo 
	anterior. Sea $W_{min}$, el costo total del \emph{Steiner Tree} 
	óptimo $T_{min}$, y sea $l$ el número de hojas en $T_{min}$.
	Entonces,
	$$\frac{W}{W_{min}}
	\le 2*\parenthize{1-\frac{1}{l}}
	\le 2*\parenthize{1-\frac{1}{\abs{S}}}$$
\end{theorem}

\begin{proof}
	Por el lema anterior, hay un ciclo $c$ en $T_{min}$ tal que por 
	(i), toda arista de $T_{min}$ aparece exactamente dos veces en 
	el ciclo y por (ii), toda hoja de $T_{min}$ aparece exactamente 
	una vez en la secuencia y si $u_i$, $u_j$ son dos hojas
	\emph{consecutivas} en el ciclo, entonces el subcamino de $u_i$ 
	a $u_{i+1}$ es simple. Podría pensarse entonces en $c$ como la 
	composición de $l$ subcaminos simples conectando una hoja a la 
	siguiente. Eliminando el subcamino simple más largo de $c$ 
	obtenemos un camino $p$ tal que, su distancia no es mayor que
	$(1-\frac{1}{l})*w{c}$, dado que el subcamino eliminado es de 
	longitud al menos $\frac{1}{l}$, como toda arista de $T_{min}$ 
	aparece exactamente dos veces en $c$, y el camino eliminado no 
	repite aristas, entonces en $p$ cada arista aparece al menos 
	una vez.
							
	Sea $v_1,v_2,\dots,v_k$ los $k=\abs{S}$ vértices para los que 
	se quiere formar el \emph{Steiner Tree}, en el orden en que 
	aparecen en $p$. Se tiene entonces,
	\begin{align*}
		w(p) & \le \parenthize{1-\frac{1}{l}}*w(c)      \\
		     & \le \parenthize{1-\frac{1}{l}}*2*W_{min} 
	\end{align*}
	Por otro lado, $w(p)$ es mayor o igual que el costo total de 
	cualquier árbol abarcador de $G_1$, dado que $G_1$ esta formado 
	por aristas entre los vértices $v_1,v_2,\dots,v_k$, todas con 
	costo a lo sumo iguales al costo de los caminos entre $w_i$ y 
	$w_j$ en $p$, luego,
	\begin{align*}
		W                 & \le w(T_S) 
		\le w(G_S) \le w(T_1) \le w(p) \\
		                  & \le        
		\parenthize{1-\frac{1}{l}}*2*W_{min}  \\
		\frac{W}{W_{min}} & \le        
		\parenthize{1-\frac{1}{l}}*2*W_{min}  \\
	\end{align*}
\end{proof}

\end{document}