\documentclass{article}

\usepackage{infiltration}

\begin{document}

\title{\textbf{Fibonacci Words}}
\author{Jorge Mederos Alvarado}
\date{}
												
\maketitle
								
\begin{statement}
	Se desea infiltrar agentes en las células de una organización, de manera que 
	se tome control sobre esta. Esta organización (\emph{ACM}) tiene una 
	estructura de comando bien compleja, lo cual hace la infiltración bastante 
	complicada.
									
	La estructura de comando del ACM consiste en varias células. Para cada par 
	de células A y B, o A controla a B, o viceversa. Pero esta relación de 
	"control" puede ser cíclica, es decir, A puede controlar B, quien a su vez 
	puede controlar C y este último controlar a A.
									
	Posemos mandar agentes a infiltrarse en cualquier célula en particular, lo 
	cual nos da control sobre esta célula y aquellas a las que controle esta, 
	pero sobre más ninguna. En el ejemplo anterior, infiltrarse en A daría 
	control sobre B pero no sobre C.
									
	Para una infiltración exitosa debemos tener control sobre todas sus células. 
	Pero se desea invertir la menor cantidad de recursos posibles, luego su 
	misión es averiguar cual es la menor cantidad de células que necesitan 
	infiltrarse para tomar control sobre la organización.
																						
	\paragraph*{Entrada.} La primera línea de cada caso de prueba contendrá un 
	entero n ($0\le n\le 75$) con el número de células. Cada una de las 
	siguientes $n$ líneas contiene un string binario de longitud $n$, donde el 
	i-ésimo caracter de la j-ésima línea nos dirá si la célula $j$ controla a la 
	célula $i$. El i-ésimo caracter de la i-esima línea es 0, y exactamente uno 
	entre en i-ésimo de la j-ésima linea y el j-esimo de la i-ésima línea es 1.
	
	\paragraph*{Salida.} Para cada caso de prueba imprima una línea que contenga 
	el número $m$ del caso, seguido por el menor número de células que deben ser 
	infiltradas para tomar control. Luego muestre $m$ números, mostrando la 
	lista de células que debe infiltrarse.
																	    
	\smallskip
	\begin{center}
		\begin{tabular}{ | m{13em} | m{13em} | } 
			\hline
			Ejemplos de Entrada & Ejemplos de Salida \\
			\hline
			2                   & Case 1: 1 2        \\ 
			00                  & Case 2: 2 1 2      \\ 
			10                  & Case 3: 2 2 3      \\ 
			3                   &                    \\ 
			010                 &                    \\ 
			001                 &                    \\ 
			100                 &                    \\ 
			5                   &                    \\ 
			01000               &                    \\ 
			00011               &                    \\ 
			11001               &                    \\ 
			10100               &                    \\ 
			10010               &                    \\ 
			\hline
		\end{tabular}
	\end{center}
\end{statement}
\bigskip

\section*{Solución}

Es posible observar que el grafo que representa la estructura de la organización es 
un torneo. Y estamos buscando un conjunto minimal, tal que para todo vértice $v$ del 
grafo que no pertenezca a ese conjunto, exista una arista desde uno de los vértices 
del conjunto hacia $v$. Esto es equivalente a encontrar un Conjunto Dominante (\emph
{Dominating Set}) minimal en el grafo torneo. Este problema puede mostrarse que es 
$np-completo$, luego es improbable que se pueda encontrar un algoritmo eficiente 
para resolverlo. Luego pudiera intuirse que será necesario realizar una búsqueda 
exhaustiva sobre los posibles conjuntos dominantes para encontrar el minimal. Sin 
embargo, como se verá a continuación, es posible diseñar un algoritmo greedy que 
minimice suficiente espacio de búsqueda, como para que enumerar todas las soluciones 
sea factible.

\begin{theorem}
	En todo torneo $T$ de a lo sumo $2^{n+1}-2$ vértices, existe un 
	conjunto dominante de tamaño $n$.
\end{theorem}

\begin{proof}
	Procedamos por inducción. Para $n=1$, cualquier torneo de $2^{n+1}
	-2=2$ vértices, tiene un vértice que domina al otro. Suponga que para 
	$n=k$ se cumple, probemos para $n=k+1$. Dado que $T$ es un torneo, se 
	cumple que,
	$$\sum_{v\in V(T)} outdeg(v) = \frac{(2^{n+1}-2)(2^{n+1}-3)}{2}$$
	luego el vértice $u$ de mayor outdegree cumple $outdeg(u)$ debe ser 
	al menos $\frac{(2^{n+1}-3)}{2}$, luego al menos $2^n-1$. Quitemos 
	dicho vértice junto con todos los vértices a los que el domina. El 
	grafo resultante es fácil ver que es un torneo. Luego la cantidad de 
	vértices restantes en el grafo es de
	\begin{equation*}
		2^{n+1}-2-1-(2^n-1) = 2^n - 2
	\end{equation*}
	Por hipótesis de inducción existe un conjunto de tamaño $n-1$ que 
	domina el grafo resultante. Si a ese conjunto añadimos el vértice que 
	eliminamos, obtenemos un conjunto dominante de tamaño $n$ para el 
	grafo original.
	
\end{proof}

Observe que dado que nuestro grafo tiene a lo sumo $75\le 2^{6+1}-2=128$, entonces 
en este se puede encontrar un conjunto dominante de tamaño a lo sumo $6$. El 
algoritmo que diseñaremos para obtener este conjunto dominante se basará en el 
procedimiento seguido para demostra el teorema anterior. Una vez obtenido un 
conjunto dominante de cardinalidad $6$ o menor, se explorarán todos los conjuntos de 
cardinalidad a lo sumo la del obtenido, para tratar de mejorar el conjunto ya 
obtenido. De esta manera disminuiríamos considerablemente el espacio de esta 
búsqueda.

\inputminted[firstline=13,linenos=false]{python}{code.py}

\begin{theorem}
	El algoritmo anterior tiene costo
	$$O\parenthize{n^2log(n)\sum_{i=1}^{log(n)-1}i^2 \binom{n}{i}}$$
\end{theorem}

\begin{proof}
	El paso greedy del agoritmo se puede asegurar que converge en $O(log(n))$ 
	iteraciones, dado que en cada iteración toma de los vértices que no son
	dominados, el de mayor outdegree. Este tiene un outdegree de al menos
	$\ceil{\frac{n-1}{2}}$, luego de eliminarlo y a todos los dominados por el,
	quedan en el grafo a lo sumo $\ceil{\frac{n-1}{2}}$ vértices, dado que en
	cada paso se reduce la cantidad de vértices por dominar a la mitad, entonces en 
	$O(log(n))$ pasos todos los vértices son dominados. Hay un factor de $n$ 
	operaciones en cada iteración que esta dado por buscar el vértice con mayor
	outdegree. Luego se obtendría un costo de $O(nlog(n))$.

	El paso de búsqueda exhaustiva tiene que buscar para todas las combinaciones de 
	tamaño $i$ de los $n$ vértices, con $k=1,2,\dots,k-1$, donde $k$ es la 
	cardinalidad del conjunto dominante encontrado por el greedy, generar cada 
	combinación tiene costo $i$ y son $\sum_{i=1}^{k-1}\binom{n}{i}$. Para cada 
	combinación, analizar si esta es un conjunto dominante nos toma $O(i*n)$ 
	iteraciones. Luego este paso tendría costo,
	$$O\parenthize{n\sum_{i=1}^{log(n)-1}i^2 \binom{n}{i}}$$

\end{proof}


\end{document}
