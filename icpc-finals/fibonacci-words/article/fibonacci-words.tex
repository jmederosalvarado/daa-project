\documentclass{article}

\usepackage{fibonacci-words}

\begin{document}

\begin{titlepage}
	\title{\textbf{Fibonacci Words}}
	\author{Jorge Mederos Alvarado}
	\date{}
								
	\maketitle
				
	\begin{statement}
		La secuencia de palabras de Fibonacci se define como
		\begin{equation*}
			F(n)=
			\begin{cases}
				0               & \text{si $n = 0$}  \\
				1               & \text{si $n = 1$}  \\
				F(n-1) + F(n-2) & \text{si $n\ge 2$} 
			\end{cases}
		\end{equation*}
		Aquí $+$ denota la concatenación de strings. El problema 
		consiste en dado un patrón $p$ y un entero $n$, cuantas veces aparece el 
		patrón en $F(n)$.
															
		\paragraph*{Entrada.} La primera línea de cada caso de prueba contendrá un 
		entero n ($0\le n\le 100$). La segunda línea contiene el patrón $p$. EL 
		patrón es no vacío y tiene una longitud de a lo sumo $10^5$ caracteres.
															
		\paragraph*{Salida.} Para cada caso de prueba imprima una línea que contenga 
		el número del caso, seguido por el número de ocurrencias del patrón $p$ en
		$F(n)$. Las ocurrencias pueden solaparse. El número de ocurrencias será 
		menor que $2^{63}$.
								    
		\smallskip
		\begin{center}
			\begin{tabular}{ | m{13em} | m{13em} | } 
				\hline
				Ejemplos de Entrada & Ejemplos de Salida          \\
				\hline
				6                   & Case 1: 5                   \\ 
				10                  & Case 2: 8                   \\ 
				7                   & Case 3: 4                   \\ 
				10                  & Case 4: 4                   \\ 
				6                   & Case 5: 7540113804746346428 \\ 
				01                  &                             \\ 
				6                   &                             \\ 
				101                 &                             \\ 
				96                  &                             \\ 
				10110101101101      &                             \\ 
				\hline
			\end{tabular}
		\end{center}
	\end{statement}
\end{titlepage}

\section*{Solución}

La primera observación que se debe hacer es que no es factible 
guardar en memoria la cadena $F(n)$, dado que para $n=100$ la 
cadena tiene tamaño $fib(n)$. Luego es necesario buscar otra 
alternativa. Es posible observar sin embargo que la cantidad de 
ocurrencias del patrón en $F(n)$ es igual a la cantidad de 
ocurrencias del mismo en $F(n-1)$, más la cantidad de ocurrencias 
en $F(n-2)$ sumándo además las ocurrencias que tienen una parte en 
una cadena y una parte en la otra.

Luego, note que el rango donde pueden estar estas ocurrencia que 
tienen parte en ambas cadenas es en los últimos $m = \abs{p}-1$ 
caracteres de $F(n-1)$ y los primeros $m$ caracteres de $F(n-2)$. 
Luego este rango si es posible guardarlo en memoria, es decir 
necesitamos los prefijos y sufijos de tamaño $m$ de estas cadenas.

\begin{theorem}
	Denótese la cantidad de ocurrencias del patrón $p$ en una 
	cadena $s$ como $P(s)$, y $Suf_k(s), Pref_k(s)$ como el sufijo 
	o prefijo, respectivamente, de tamaño $k$ de la cadena $s$.
	\begin{align*}
		P(F(n)) & = P(F(n-1)) + P(F(n-2)) + P(C)       \\
		C       & = {Suf}_m(F(n-1)) + {Pref}_m(F(n-2)) \\
	\end{align*}
\end{theorem}

\begin{proof}
	Es fácil ver que toda ocurrencia del patrón $p$ en $F(n)$, está 
	o completamente contenida en $F(n-1)$, completamente contenida 
	en $F(n-2)$ o tiene partes en ambos lados. Si $p$ tiene partes 
	en ambos lados entonces un prefijo de $p$ ocurre en $F(n-1)$ y 
	un prefijo de $p$ ocurre en $F(n-2)$, como $p$ tiene al menos 
	un caracter en $F(n-1)$ y uno en $F(n-2)$ entonces $p$ tiene a 
	lo sumo $m$ caracteres en $F(n-1)$ y a lo sumo $m$ caracteres 
	en $F(n-2)$. Luego cada ocurrencia compartida de $p$ a ambos 
	lados, necesariamente ocurre en la cadena formada por
	$Suf_m(F(n-1))$ concatenada con $Pref(F(n-2))$. De igual manera 
	cada ocurrencia de $p$ en esta cadena implicaría una ocurrencia 
	en $F(n)$ que tiene caracteres a ambos lados.
			
\end{proof}

\inputminted[firstline=10, linenos=false]{python}{code.py}

\begin{theorem}
	El algoritmo anterior tiene complejidad temporal
	$O(n*\abs{p})$.
\end{theorem}

\begin{proof}
	El costo del algoritmo anterior está dado por el costo del 
	ciclo, este realiza $n-1$ iteraciones, en cada una de estas 
	iteraciones se realiza un llamado a \emph{kmp} para calcular la 
	cantidad de ocurrencias del patrón en la cadena $center$, la 
	cual tiene longitud a lo sumo $2(\abs{p}-1)$, por lo que este 
	cómputo tendría costo $O(\abs{p})$. Se realizan dentro del 
	ciclo también operaciones de concatenación entre strings, las 
	cuales dado que dichos strings tienen a lo sumo longitud
	$\abs{p}-1$, pueden tener costo hasta $O(\abs{p})$. Por tanto 
	el ciclo tendría costo $O(n*\abs{p})$ y también lo tendría el 
	algoritmo anterior.
			
\end{proof}

\begin{note}
	Durante el análisis anterior se considera que el costo del algoritmo \emph{kmp} 
	para encontrar la cantidad de ocurrencias de un patrón $p$ en un texto $s$ es
	$O(\abs{s}+\abs{p})$.
\end{note}

\end{document}
