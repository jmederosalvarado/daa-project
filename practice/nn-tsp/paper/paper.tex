\documentclass[spanish]{llncs}

\usepackage[utf8]{inputenc}

\usepackage{amsmath}

\usepackage{xcolor}
\definecolor{backcolour}{rgb}{0.95,0.95,0.97}

\usepackage{minted}
\usemintedstyle{friendly}
\setminted {
	linenos = true,
	bgcolor = backcolour
}

\usepackage{dsfont}
\newcommand*{\Z}{\mathds{Z}}

\begin{document}

\title{Aproximación al Problema del Viajante con Algoritmo Vecino más Cercano}

\author{Jorge Mederos}
\authorrunning{J. Mederos}

\institute{Universidad de La Habana}

\maketitle

\section{El Problema}

El problema del viajante es un problema bastante clasico, la idea consiste
en dadas $n$ ciudades, y un costo de viajar de una ciudad a otra, encontrar el
recorrido que pasa por todas las ciudades y tiene costo mínimo.

La entrada es un grafo $K_n$ donde para cada arista $(i,j)$ existe $d(i,j)$,
distancia entre los nodos $i$ y $j$.

El problema del viajante metrico, es una variante del problema anterior que cumple
que la funcion $d$ es una metrica, es decir cumple con la desigualdad triangular,
es simetrica y $d(i,i)=0$.

\paragraph*{Fuente.}Este problema y su solucion se puede encontrar
en el articulo \emph{AN ANALYSIS OF SEVERAL HEURISTICS FOR THE TRAVELING
SALESMAN PROBLEM} de los autores \emph{Daniel J. Rosenkrantz}, \emph{Richard E. Stearns}
y \emph{Phillip M. Lewis}.

\section{Algoritmo del Vecino mas Cercano}

La idea de este algoritmo es bastante sencilla

\begin{enumerate}
	\item Comienza en un nodo arbitrario
	\item Elige el nodo mas cercano al ultimo seleccionado, y que todavía no forme 
	      parte del camino
	\item Cuando todos los nodos se hayan añadido al camino, añada una arista del
	      ultimo nodo al primero
\end{enumerate}

Se asume que si hay empates en el paso 2, estos se resuelven arbitrariamente.

Note que este algoritmo puede ser programado en $O(n^2)$, donde $n$ es la
cantidad de nodos, esto es linear en el tamaño de la entrada si se asume que la
entrada es una lista de todas las distancias.

\inputminted{python}{code.py}

Sea $NN$ la longitud del recorrido obtenido por el algoritmo del Vecino mas Cercano
y $OPT$ la longitud del recorrido optimo.

\begin{lemma}
	\label{lemma:1}
	Supongamos que para un grafo de $n$ nodos $(N,d)$ existe una forma de asignarle
	a cada nodo $p$ un numero $l_p$ tal que las siguientes dos condiciones se cumplan:
	\begin{enumerate}
		\item $d(p,q) \ge min(l_p,l_q)$ para todo nodo p y q.
		\item $l_p \le \frac{1}{2}*OPT$ para todo nodo p.
	\end{enumerate}
	Entonces 
	$$\sum_{p \in N} l_p \le \frac{1}{2}(\lceil log_2(n) \rceil + 1)$$
\end{lemma}

\begin{proof}
	Podemos asumir, sin pérdida de generalidad, que el conjunto de nodos
	$N = \{i \vert 1 \le i \le n\}$ y que $l_i \ge l_j$ siempre que $i \le j$.
	La clave es la prueba de la siguiente desigualdad:
	\begin{equation}
		\label{eq:lemma:key}
		OPT \ge 2 * \sum_{i=k+1}^{min(2k,n)} l_i \qquad \forall_{k, 1 \le k \le n}
	\end{equation}

	Para probar \eqref{eq:lemma:key}, sea H el subgrafo completo definido sobre
	el conjunto de nodos $\{i \vert 1 \le i \le min(2k,n)\}$.

	Sea $T$ el recorrido de $H$ que visita los nodos de $H$ en el mismo orden
	que los nodos son visitados en el recorrido optimo del grafo original. Sea
	$LEN$ la longitud de $T$. Por la desigualdad triangular, cada arista $(b,c)$
	de $T$ debe tener una longitud que es menor o igual que el camino de $b$ a $c$
	usado en el recorrido optimo. Dado que las aristas de $T$ suman $LEN$ y los
	caminos correspondientes en el grafo original suman $OPT$, podemos concluir que
	\begin{equation}
		\label{eq:opt-len}
		OPT \ge LEN
	\end{equation}

	Por la primera condición del lema~\ref{lemma:1}, por cada $(i,j)$ en $T$,
	$d(i,j) \ge min(l_i,l_j)$, por tanto,
	\begin{equation}
		\label{eq:len-sums}
		LEN \ge \sum_{(i,j) \in T} min(l_i,l_j) = \sum_{i \in H} a_il_i
	\end{equation}
	donde $a_i$ es el numero de aristas $(i,j)$ en $T$ para las cuales $i>j$
	(y por tanto $l_i = min(l_i,l_j)$).

	Observese que cada $a_i$ es a lo sumo 2 (porque $i$ es extremo de solo
	2 aristas en el recorrido) y la suma de los $a_i$ es el numero de aristas
	en $T$. Como $k$ es al menos la mitad del numero de aristas en $T$, si asumimos
	que los $k$ $l_i$ mas grandes tienen $a_i=0$ y los restantes $min(2k,n)-k$ valores
	$l_i$ tienen $a_i=2$, entonces:
	\begin{equation}
		\label{eq:lb}
		\sum_{i \in H} a_il_i \ge 2 * \sum_{i=k+1}^{min(2k,n)}l_i
	\end{equation}
	
	Notese que en caso de que los $k$ mayores $l_i$ no tengan todos $a_i=0$, entonces
	dado que la suma de todas las aristas de $H$ se mantiene la misma, y $ai \le 2$,
	el lado derecho de \eqref{eq:lb} solo haría disminuir, mateniendose la desigualdad.
	Luego a partir de \eqref{eq:opt-len}, \eqref{eq:len-sums} y \eqref{eq:lb} se obtiene
	\eqref{eq:lemma:key}.

	Ahora se suma la desigualdad \eqref{eq:lemma:key} para todos los valores de $k$ iguales a
	una potencia de 2 menor que $n$, es decir:
	\begin{equation*}
		\sum_{j=0}^{\lceil log_2(n) \rceil - 1} OPT \ge \sum_{j=0}^{\lceil log_2(n) \rceil - 1} 2 * \sum_{i=2^j+1}^{min(2^{j+1},n)}l_i
	\end{equation*}
	la cual se reduce a:
	\begin{equation}
		\label{eq:sum}
		\lceil log_2(n) \rceil * OPT \ge 2*\sum_{i=2}^n l_i
	\end{equation}

	La segunda condicion del lema~\ref{lemma:1} implica $OPT \ge 2*l_1$
	la cual combinada con \eqref{eq:sum} dan conclusión al lema~\ref{lemma:1}.
\end{proof}

\begin{theorem}
	\label{theo:1}
	Para un grafo del problema del viajante con $n$ nodos:
	\begin{equation}
		\frac{NN}{OPT} \le \frac{1}{2}*(\lceil log_{2}(n) \rceil + 1)
	\end{equation}
\end{theorem}

\begin{proof}
	Para cada nodo $p$, sea $l_p$ la distancia de la arista que va del nodo $p$
	al proximo nodo seleccionado por el algoritmo como vecino mas cercano a $p$.

	Si el nodo $p$ fue seleccionado primero que el nodo $q$, entonces $q$ fue un
	candidato a ser el vecino disponible mas cercano de $p$. Esto significa que la
	arista $(p,q)$ no es mas corta que la arista seleccionada y por tanto:
	\begin{equation}
		\label{eq:lemma:cond-a1}
		d(p,q) \ge l_p
	\end{equation}
	
	De lo contrario, si $q$ fue seleccionado primero que $p$:
	\begin{equation}
		\label{eq:lemma:cond-a2}
		d(p,q) \ge l_q
	\end{equation}
	
	Como uno de los nodos fue seleccionado primero que el otro, debe cumplirse alguna de
	las ecuaciones \eqref{eq:lemma:cond-a1} o \eqref{eq:lemma:cond-a2} y por tanto se cumple
	la primera condicion del lema~\ref{lemma:1}.

	Para demostrar la segunda condicion, basta probar que sea la arista $(p,q)$
	\begin{equation}
		\label{eq:lemma:cond-b}
		d(p,q) \le \frac{1}{2}*OPT
	\end{equation}
	
	El recorrido optimo se puede considerar que consiste de dos partes disjuntas, cada una de
	ellas es un camino entre los nodos $p$ y $q$. De la desigualdad triangular, la longitud
	de cualquier camino entre $p$ y$q$ no puede ser menor que $d(p,q)$, cumpliéndose asi
	\eqref{eq:lemma:cond-b}.

	Como los $l_p$ son las longitudes de las aristas que comprenden el recorrido,
	entonces:
	\begin{equation}
		\label{eq:lp-nn}
		\sum_{p \in N} l_p = NN
	\end{equation}

	La conclusión del lema~\ref{lemma:1} junto con \eqref{eq:lp-nn} implican el cumplimiento del
	teorema~\ref{theo:1}.
\end{proof}

\end{document}