\documentclass[spanish]{llncs}

\usepackage[utf8]{inputenc}

\usepackage{amsmath}

\usepackage{xcolor}
\definecolor{backcolour}{rgb}{0.95,0.95,0.97}

\usepackage{minted}
\usemintedstyle{friendly}
\setminted {
	linenos = true,
	bgcolor = backcolour
}

\usepackage{dsfont}
\newcommand*{\Z}{\mathds{Z}}

\begin{document}

\title{K-Agrupamiento de espaciamiento maximo}

\author{Jorge Mederos}
\authorrunning{J. Mederos}

\institute{Universidad de La Habana}

\maketitle

\section{El Problema}

\paragraph*{Fuente.} Este problema fue extraído del libro \emph{Algorithm Design}
de los autores \emph{Jon Kleinberg} y \emph{Eva Tardos} en el \emph{epígrafe 4.7}.

Suponga que nos son dados un conjunto $U$ de $n$ objetos, etiquetados
$p_1$, $p_2$, $\dots$, $p_n$. Para cada par, $p_i$ y $p_j$, tenemos una
distancia numerica $d(p_i,p_j)$ que cumple:
\begin{enumerate}
	\item $d(p_i,p_i) = 0$
	\item $d(p_i,p_j)>0$ para $p_i \neq p_j$
	\item $d(p_i,p_j) = d(p_j,p_i)$
\end{enumerate}

Suponga que buscamos dividir los objetos de $U$ en $k$ grupos, para
un parámetro $k$. Decimos que un \emph{k-agrupamiento} de $U$ es una particion
de $U$ en $k$ conjuntos no vacios. Definimos el espaciamiento de un
\emph{k-agrupamiento} como la menor distancia entre cualquier par de puntos
de grupos diferentes. Se desea encontrar el \emph{k-agrupamiento} de espaciamiento
maximo.

\section{Solución}

Para encontrar el agrupamiento de espaciamiento maximo, consideramos un grafo
que tiene como vertices los elementos de $U$. Las componentes conexas seran los
grupos, y uniremos los puntos cercanos lo mas rapido posible (de esta manera,
no terminan en grupos diferentes elementos muy cercanos entre ellos). Luego,
en cada iteración pondremos una arista en el grafo que una el par de vertices
mas cercanos (si estos no estan ya en la misma componente conexa). Detendremos
nuestro algoritmo cuando tengamos $k$ componentes conexas.

Note que la idea anterior es muy similar al algoritmo de Kruskal para hallar el
árbol abarcador de costo minimo de un grafo. El algoritmo anterior es equivalente
a ejecutar el algoritmo de Kruskal sobre un grafo $G$ definido sobre $U$, que tiene
una arista de costo $d(p_i,p_j)$ entre cada par de vertices $p_i,p_j$. La unica
diferencia es que este es detenido luego de tener $k$ componentes conexas dado que
se busca un \emph{k-agrupamiento}. Es decir no se añaden las $k-1$ ultimas aristas
que añade el algoritmo de Kruskal.

\inputminted{python}{code.py}


\begin{theorem}
	Las componentes conexas $C_1,C_2,\dots,C_k$ formadas de eliminar
	las $k-1$ aristas mas costosas del árbol abarcador de costo minimo
	creado por el algoritmo de Kruskal, constituyen un \emph{k-agrupamiento}
	de espaciamiento maximo.
\end{theorem}

\begin{proof}
	Denotese por \emph{K} el arbol abarcador de costo minimo que devuelve
	el algoritmo de Kruskal y por \emph{C} el agrupamiento $C_1,C_2,\dots,C_k$.
	El espaciamiento de \emph{C} es precisamente la longitud $d^*$ de la
	\emph{k-1} arista mas costosa de \emph{K}, esta es la proxima arista que
	hubiera añadido el algoritmo de Kruskal de no haber sido detenido.

	Considerese otro \emph{k-agrupamiento} $C'$, que particiona $U$ en conjuntos
	no vacios $C'_1,C'_2,\dots,C'_k$. Debe probarse que el espaciamiento de $C'$
	es a lo sumo $d^*$.

	Dado que los agrupamientos son diferentes, debe ser que hay un par de puntos
	$p_i,p_j \in C_r$ y $p_i \in C'_s$, $p_j \in C'_t$ con $C'_s \neq C'_t$.

	Dado que $p_i$ y $p_j$ pertenecen a la misma componente $C_r$, debe ser que el
	algoritmo de Kruskal añadio las aristas de un camino $P$ de $p_i$ a $p_j$ antes
	de ser detenido. En particular, esto significa que cada arista en $P$ tiene a lo
	sumo longitud $d^*$. Ahora, se conoce que $p_i \in C'_s$ pero $p_j \notin C'_s$;
	sea $p'$ la primera arista de $P$ que no pertenece a $C'_s$, y sea $p$ el nodo
	de $P$ que viene justo antes de $p'$. Como se expresó anteriormente,
	$d(p,p') \le d^*$, dado que la arista $(p,p')$ fue añadida por el algoritmo
	de Kruskal. Pero $p$ y $p'$ pertenecen a diferentes conjuntos en el agrupamiento
	$C'$, y por tanto el espaciamiento de $C'$ es a lo sumo $d(p,p') \le d^*$. Con esto
	queda demostrado el teorema anterior.
\end{proof}

\end{document}