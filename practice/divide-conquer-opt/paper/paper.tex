\documentclass[spanish]{llncs}

\usepackage[utf8]{inputenc}

\usepackage{amsmath}

\usepackage{xcolor}
\definecolor{backcolour}{rgb}{0.95,0.95,0.97}

\usepackage{minted}
\usemintedstyle{friendly}
\setminted {
	linenos = true,
	bgcolor = backcolour
}

\usepackage{dsfont}
\newcommand*{\Z}{\mathds{Z}}

\usepackage{hyperref}

\begin{document}

\title{Ordenando montículos}

\author{Jorge Mederos}
\authorrunning{J. Mederos}

\institute{Universidad de La Habana}

\maketitle

\section{El Problema}

Una compañía de minería extrae metal de los alrededores de un río.
Las minas estan colocadas en $n$ puntos a lo largo del río, cada uno
identificado por la distancia desde el cauce del río. En cada mina se
mantiene un montículo de mineral extraído.

Para recolectar el mineral, la compañía reagrupa los $n$ montículos en $k$
de los puntos de mina, donde luego son recogidos por camiones.

Para transportar los montículos, se utiliza una barca, que permite en la
práctica cargar cualquier cantidad de mineral. La barca sale del comienzo del
río y solo viaja en una dirección. Los montículos o bien son completamente movidos
de punto o se mantienen en su mina. El costo de mover un montículo de peso $W$ desde
un punto $X$ a un punto $Y$ es $W*(Y-X)$. El costo total de reagrupar es la suma del
costo por cada movimiento.

Dados los valores de $W_1, W_2, \dots, W_n$ y un entero $k$, diga el menor costo
de reagrupar los montículos.

\paragraph*{Fuente.} Este problema fue extraído del sitio 
\href{https://onlinejudge.org/index.php?option=com_onlinejudge&Itemid=8&page=show_problem&problem=3969}{online judge}

\section{Solucion}

Definamos $c(i,j)$ como el menor costo de agrupar los montículos de las primeras $j$ minas
en $j$ de ellas. Luego se obtendría la siguiente recuerrencia:
\begin{align}
	\label{eq:rec:c}
	c(i,j)                          & =                          
	\begin{cases}
	w(1,j)                          & \text{ if $i = 1$}         \\
	\min_{k<j}(c(i-1,k) + w(k+1,j)) & \text{ if $i > 1$}         \\
	\end{cases}
	\\
	\label{eq:rec:w}
	w(i,j)                          & = \sum_{k=i}^{j} W_k*(j-k) 
\end{align}

\begin{theorem}
	La funcion $c(i,j)$ calcula el menor costo de agrupar los monticulos de
	las primeras j minas en i de ellas.
\end{theorem}

\begin{proof}
	$c(1,j)$ es calculada correctamente, dado que el menor costo posible para
	reagrupar los $j$ primeros monticulos en una mina, es precisamente de trasladar
	cada monticulos hasta la ultima mina. Es facil mostrar por inducción que si
	$c(t,j)$ se calcula de manera correcta, entonces tambien lo hace $c(t+1,j)$.
\end{proof}

\inputminted{python}{basic.py}

\section{Optimizando con Divide y Vencerás}

La tecnica de \textbf{divide y vencerás}, ademas de otras muchas aplicaciones
que posee, puede ser optimizada para optimizar la complejidad temporal de un
algoritmo de programación dinamica, si este cumple ciertas propiedades.

Una de las propiedades que tiene que cumplir un algoritmo de programación
dinámica para ser optimizado con esta tecnica es que la recurrencia sea de
la forma $dp(i,j) = \min_{k<j}(dp[i-1][k] + w(k+1,j))$ para alguna funcion
de costo $w$.

\begin{property}
	\label{prop}
	Sea $K(i,j)$ el valor $k$ para el cual $dp(i,j) = dp[i-1][k] + w(k+1,j)$,
	para cualquier $i,j$ se cumple que $K(i,j) \le K(i,j+1)$, esto significa que
	el óptimo $k$ es monótono en $j$ para un $i$ fijo.
\end{property}

Como se pudo apreciar en la seccion anterior, este tipo de recurrencias es posible
calcularlas en $O(m*n^2)$, con esta tecnica el costo de calcular la recurrencia
sería $O(m*n*log(n))$.

El algoritmo para calcular $dp[i][j]$ para un $i$ fijo y todo $j$ se presenta a
continuación.

\inputminted{python}{optimization.py}

Este algoritmo se aprovecha de la propiedad~\ref{prop} para en cada llamado
recursivo calcular el nuevo intervalo en el que estan las posibles soluciones.
Descomponiendo cada rango de $j$ en 2 intervalos que tienen longitud la mitad
del original. Logrando asi un costo $O(n*log(n))$

El algoritmo original luego de aplicarle esta optimizacion quedaría de la siguiente
manera.

\inputminted{python}{optimized.py}

\subsection{Desigualdades cuadrangulares}

\begin{definition}
	\label{def:qi}
	Sean $a,b,c,d \in \Z$, donde $a \le b \le c \le d$. $f(x,y)$ satisface
	la desigualdad cuadrangular si:
	\begin{equation*}
		f(a,c) + f(b,d) \le f(a,d) + f(b,c)
	\end{equation*}
	En lo adelante se hará referencia a la desigualdad cuadrangular como QI.
\end{definition}

\begin{theorem}
	Sea $K(i,j)$ el menor $k$ tal que $dp(i,j) = dp(i,k-1) + c(k,j)$.
	Si $K$ cumple $QI$ entonces $K(i,j) < K(i,j')$ para $j' >= j$.
\end{theorem}

\begin{proof}
	Para $x=K(s,i)$, $y=K(s,j)$ tal que $x \le j$, $y \le i$ y $x \neq y$
	pruebese que:
	\begin{equation}
		w(x,i) + w(y,j) < w(x,j) + w(y,i)
	\end{equation}

	$dp(s,i) = dp(s-1,x) + w(x,i)$ $\implies$ $w(x,i) = dp(s,i) - dp(s-1,x)$,
	analogamente $w(y,j) = dp(s,j) - dp(s-1,y)$, por tanto basta mostrar que						
	\begin{equation}
		\begin{split}
			dp(s,i) - dp(s-1,x) + dp(s,j) - dp(s-1,y) &< w(x,j) + w(y,i) \\
			dp(s,i) + dp(s,j) &< dp(s-1,x) + w(x,j) + dp(s-1,y) + w(y,i)
		\end{split}
	\end{equation}
	
	Por definición, $dp(s,j) \le dp(s-1,x) + w(x,j)$ y
	$dp(s,i) \le dp(s-1,y) + w(y,i)$. Sin embargo, note que solo
	una de esta igualdades se sostiene. Si la primera es verdadera,
	entonces $x>y$, dado que si $x < y$ entonces $y=K(s,j)$ ya no es el menor
	valor tal que $dp(s,j) \le dp(s-1,x) + w(x,j)$. De la misma manera si la
	segunda fuera verdadera, se tendría $x<y$, lo cual contradice la primera.
	Luego, se obtiene:
	\begin{equation*}
		dp(s,j)+dp(s,i)<dp(s-1,x)+w(x,j)+dp(s-1,y)+w(y,i)
	\end{equation*}
	como es deseado.

	Supongase entonces que para $i<j$, $K(s,i)>K(s,j)$. Sea $x=K(s,i)$ y 
	$y=K(s,j)$. Por tanto, como $y < x < i < j$, por lo demostrado anteriormente
	se obtiene $w(x,i) + w(y,j) < w(x,j) + w(y,i)$. Lo cual contradice QI. Luego
	para $i<j$, $K(s,i) \le K(s,j)$.
\end{proof}

Es facil verificar que $w$~\eqref{eq:rec:w} cumple QI, luego $dp$ puede ser
optimizada mediante la tecnica de \textbf{divide y vencerás} antes vista.

\end{document}